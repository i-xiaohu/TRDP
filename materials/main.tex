\documentclass[12pt, a4paper, oneside]{article}
\usepackage[UTF8]{ctex}
\usepackage{titlesec}
\usepackage{enumitem}
\usepackage{amsmath, amsthm, amssymb}
\usepackage{graphicx}
\usepackage{hyperref}
\hypersetup{
    colorlinks=true,     % Enable colored links
    linkcolor=blue,      % Color for normal internal links
    citecolor=black,     % Color for citations
    filecolor=black,     % Color for file links
    urlcolor=black,      % Color for URLs
    pdfborder={0 0 0}    % No border around links
}
\usepackage{geometry}
\geometry{
    top=1.5cm,
    bottom=1.5cm,
    left=2cm,
    right=2cm
}

\titleformat{\section}{\normalfont\Large\bfseries}{\thesection}{1em}{}
\title{\textbf{包含拷贝事件的动态规划序列比对}}
\author{季发虎}
\date{2025年12月20日}

\begin{document}
\maketitle

\section{两阶段的动态规划模型}
Gary Benson的DSI(Duplication Substitution Indel)[\ref{gary_1997}] 模型是首个可以拟合生物序列中拷贝变异事件的序列比对模型。给定序列 $A$ 和 $B$,DSI模型的状态递推式为:

\[
\left\{
\begin{aligned}
    E_{i,j} & = \max(E_{i-1,j}, H_{i-1,j} + G_o) + G_e \\
    F_{i,j} & = \max(F_{i,j-1}, H_{i,j-1} + G_o) + G_e \\
    M_{i,j} & = \max(H_{i-1,j-1} + \text{mat}(A[i], B[j])) \\
    D_{i,j} & = \max(H_{i-x,j-y} + \text{dup}(A[i-x:x], B[j-y:y]) + G_e) \\
    H_{i,j} & = \max(E_{i,j}, F_{i,j}, M_{i,j}, D_{i,j})
\end{aligned}
\right.
\]

其中 $\text{dup}(a, b])$ 表示将序列 $a^{*}$ 比对至序列 $b$,$a^{*}$ 代表 $a$ 重复任意次数。该函数实现方法为全局的wrapround动态规划[\ref{wraparound}],时间复杂度为 $O(|a||b|)$。

若序列 $A$ 和 $B$ 的长度分别为 $N$ 和 $M$,则DSI模型的时间复杂度为 $O(N^2M^2)$。鉴于该模型的低扩展性,本项目旨在提出一个平方级时间复杂度的解决方案,包含两个步骤:

\begin{enumerate}[itemsep=0pt, topsep=0pt, label=(\arabic*)]
    \item 对序列进行自比对,以识别序列自身所包含的复制事件。
    \item 在上一阶段中识别出的重复断点处进行内循环,做wraparound DP来插入复制事件。
\end{enumerate}

该方案假设序列之间的拷贝数变异仅发生于序列内部的重复断点处。这一假设大幅减少了DSI模型中允许发生重复事件的状态数,有机会使DSI模型在真实长序列上具备实用性。

然而,化简模型必然会损失原始理论的一般性,尤其体现在第一阶段。第一阶段的任务是寻找序列自身的重复事件,然而序列自身的重复存在多种表示形式,当拷贝之间存在差异时,寻找最优的表示形式并不是一个简单问题。我们目前采取的方案是寻找尽可能小的重复单元。这一方面可以缓解表示歧义,另一方面可以增加拷贝断点的数量,以提升第二阶段识别潜在拷贝变异事件的灵敏度。

\section{初步实现}

\textbf{识别序列内重复事件}。令 $F_i$ 为序列 $S[1:i]$ 与自身比对的结果,其中可能包含重复事件,$F_i$ 的转移公式为:

\[
F_i = \max
\begin{cases}
    F_{i-1} + \text{mat}(S[i], S[i])) \\
    F_j + \text{dup2}(S[k:i], S[j:i-k)) + G_{open} \quad k \in [R_l, R_r], j \in [1, i-k) \\ 
\end{cases}
\]

其中,第一行代表来自对角线的自匹配转移,第二行为发生重复变异事件的转移。该方法枚举重复片段的长度(范围 $[R_l, R_r]$ 为外置参数),并尝试将重复单元 $S[k:i]$ 比对至 $S[j:i-k)$。此处的 $\text{dup2}(a,b)$ 为修改之后的global wrapround DP,主要区别在于奖励重复事件的发生。若 $a$ 重复了 $n$ 次,则 $\text{dup2}(a, b) = \text{dup}(a, b) + n G_{dup}$。其中,$G_{dup}$ 为每次复制的奖励分数(默认为1)。与之对应的 $G_{open}$ 则对应着开始这次重复事件的初始罚分(默认为-5)。该方法有如下两个作用:

\begin{enumerate}[itemsep=0pt, topsep=0pt, label=(\arabic*)]
    \item 在自比对与wraparound DP打分矩阵完全相同的情况下,驱离最优路径远离对角线。
    \item 通过奖励重复次数,该方法偏向长度更小的重复单元,避免歧义表示。
\end{enumerate}

从序列末尾回溯该动态规划即可得到序列内部所有的重复事件。若 $F_i$ 由 $F_{i-1}$ 转移而来,则该区域为非重复区域。若 $F_i$ 由 $F_j$ 转移而来,则此处一定存在一个重复事件。且该重复事件一定为最优的转移策略,因为$F_j$ 已经考虑了先前所有的重复事件。且由于 $G_{open}$ 的存在,$F_i$ 和 $F_j$ 不可能对应同一个重复事件。

假设重复单元的最大长度为 $L$,序列 $S$ 的长度为 $N$,则该方法在序列的每个位置枚举重复单元长度 $L$ 并做时间复杂度为 $O(NL)$ 的wrapround DP,总时间复杂度为 $O(N^2L^2)$。

\begin{figure}[h!]
    \centering
    \includegraphics[width=0.7 \textwidth]{stage1.png}
    \caption{自比对识别重复事件}
    \label{fig_stage1}
\end{figure}

在简单的模拟数据(重复单元5-20bp,重复次数10-25,两翼区域20-120bp,拷贝变异率 $\le 5\%$)上测试,该方法可以完全还原出重复区域与非重复区域,并且能够识别出重复单元以及次数(图\ref{fig_stage1})。

当前方法的问题在于时间复杂度高,可扩展性较差。但这主要是由 $O(N^2)$ 主导的,重复单元的长度 $L$(一般 $\le 200$ bp)理论上远小于序列长度 $N$。

\textbf{比对两条序列}。在自比对获得重复事件的断点位置后,本方法对DSI模型做以下修改。

\[
\left\{
\begin{aligned}
    E_{i,j} & = \max(E_{i-1,j}, H_{i-1,j} + G_o) + G_e \\
    F_{i,j} & = \max(F_{i,j-1}, H_{i,j-1} + G_o) + G_e \\
    M_{i,j} & = \max(H_{i-1,j-1} + \text{mat}(A[i], B[j])) \\
    D_{i,j} & = \max(H_{i-x,j-y} + \text{dup}(A[i-x:x], B[j-y:y]) + G_e) \\
    & \quad \quad i-x, j-y \in B \text{: \{break points from stage 1\}} \\
    H_{i,j} & = \max(E_{i,j}, F_{i,j}, M_{i,j}, D_{i,j})
\end{aligned}
\right.
\]

在两条包含VNTR变异的模拟序列上,图\ref{fig_stage2} 给出了DSI模型的比对结果。该算法的时间复杂度为 $O(NMLB)$,其中SI模型可以通过SIMD,Banded以及Wavefront等算法加速,内层的wrapround可以通过SIMD,Wavefront以及多核心并行加速。

\begin{figure}[h!]
    \centering
    \includegraphics[width=0.7 \textwidth]{stage2.png}
    \caption{DSI模型比对结果}
    \label{fig_stage2}
\end{figure}

\begin{figure}[h!]
    \centering
    \includegraphics[width=0.7 \textwidth]{gsw_matrix1.png}
    \caption{SI模型比对结果}
    \label{fig_sw}
\end{figure}

相比不包含重复事件的SI模型,Smith-Waterman算法给出不连续的插入和删除结果,无法真实反应拷贝数变异(图\ref{fig_sw})。

该方法在真实数据上的表现有待测试。

\section{讨论}

目前所实现的第一阶段算法与前期讨论的D转移策略并不一致,这主要是因为D转移的正确性或正确实现D转移的复杂性有待讨论。

\begin{enumerate}[itemsep=0pt, topsep=0pt, label=(\arabic*)]
    \item D转移提供的通道不具备一致性,即无法约束重复单元,识别不出与上述算法一致的最优重复。
    \item 零花费从后向前转移,导致全局路径错误。若 $(i_2, j_2)$ 从 $(i_1, j_1)$ 转移而来,矩阵中可能存在介于二者之间的 $(i_3, j_3)$ 满足 $(i_1, j_1) \rightarrow (i_3, j_3) + (i_3, j_3) \rightarrow (i_2, j_2) > (i_1, j_1) \rightarrow (i_2, j_2)$。也就是,一次计算未能达到最优,再次循环更新会产生新的结果。此处请参考wrapround论文[\ref{wraparound}],文中重点讨论了不同周期之间状态转移的正确性:第二轮更新时,只有匹配+删除才会产生更优的结果。
    \item 也许以上错误可以通过规则约束实现,但正确的约束可能并不简单,这损害了理论算法的简洁性。
\end{enumerate}

最后附上我的个人建议。第一阶段的核心目标是识别潜在的重复事件位点,该过程其实可以由时间复杂度为 $O(N\log{N})$ 的分治算法实现[\ref{gad_2001}, \ref{michael_1984}],这将大幅增加算法的效率和可优化空间。

\section{参考文献}

\begin{enumerate}
    \item Gary Benson. 1997. Sequence alignment with tandem duplication. \label{gary_1997}
    \item Vincent A. Fischetti, Gad M. Landau, Jeanette P. Schmidt and Peter H. Sellers. 1992. Identifying periodic occurrences of a template with applications to protein structure. \label{wraparound}
    \item Gad M. Landau, Jeanette P. Schmidt and Dina Sokol. 2001. An algorithm for approximate tandem repeats. \label{gad_2001}
    \item Michael G. Main and Richard J. Lorentz. 1984. An O (n log n) algorithm for finding all repetitions in a string. \label{michael_1984}
\end{enumerate}

\end{document}