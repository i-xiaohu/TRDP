\documentclass[12pt, a4paper, oneside]{article}
\usepackage[UTF8]{ctex}
\usepackage{titlesec}
\usepackage{enumitem}
\usepackage{amsmath, amsthm, amssymb}
\usepackage{graphicx}
\usepackage{hyperref}
\hypersetup{
    colorlinks=true,     % Enable colored links
    linkcolor=blue,      % Color for normal internal links
    citecolor=black,     % Color for citations
    filecolor=black,     % Color for file links
    urlcolor=black,      % Color for URLs
    pdfborder={0 0 0}    % No border around links
}
\usepackage{geometry}
\geometry{
    top=1.5cm,
    bottom=1.5cm,
    left=2cm,
    right=2cm
}
\usepackage{multirow}
\usepackage{tcolorbox}

\titleformat{\section}{\normalfont\Large\bfseries}{\thesection}{1em}{}
\title{\textbf{包含拷贝事件的动态规划序列比对}}
\author{季发虎}
\date{2025年1月6日}

\begin{document}
\maketitle

\section{两阶段的动态规划模型}
Gary Benson的DSI(Duplication Substitution Indel)[\ref{gary_1997}] 模型是首个可以拟合生物序列中拷贝变异事件的序列比对模型。给定序列 $A$ 和 $B$,DSI模型的状态递推式为:

\[
\left\{
\begin{aligned}
    E_{i,j} & = \max(E_{i-1,j}, H_{i-1,j} + G_o) + G_e \\
    F_{i,j} & = \max(F_{i,j-1}, H_{i,j-1} + G_o) + G_e \\
    M_{i,j} & = \max(H_{i-1,j-1} + \text{mat}(A[i], B[j])) \\
    D_{i,j} & = \max(H_{i-x,j-y} + \text{dup}(A[i-x:x], B[j-y:y]) + G_e) \\
    H_{i,j} & = \max(E_{i,j}, F_{i,j}, M_{i,j}, D_{i,j})
\end{aligned}
\right.
\]

其中 $\text{dup}(a, b])$ 表示将序列 $a^{*}$ 比对至序列 $b$,$a^{*}$ 代表 $a$ 重复任意次数。该函数实现方法为全局的wrapround动态规划[\ref{wraparound}],时间复杂度为 $O(|a||b|)$。

若序列 $A$ 和 $B$ 的长度分别为 $N$ 和 $M$,则DSI模型的时间复杂度为 $O(N^2M^2)$。鉴于该模型的低扩展性,本项目旨在提出一个平方级时间复杂度的解决方案,包含两个阶段:

\begin{enumerate}[itemsep=0pt, topsep=0pt, label=(\arabic*)]
    \item 对序列进行自比对,识别序列自身所包含的复制事件。
    \item 在上一阶段中识别出的重复断点处进行内循环(wraparound),以插入复制事件。
\end{enumerate}

该方案假设序列之间的拷贝数变异仅发生于序列内部的重复断点处。这一假设大幅减少了DSI模型中允许发生重复事件的状态数,有机会使DSI模型在真实长序列上具备实用性。

然而,化简模型必然会损失原始理论的一般性,尤其体现在第一阶段。第一阶段的任务是寻找序列自身的重复事件。但序列自身的重复存在多种表示形式(如嵌套),这使得重复序列的最优形式变得很复杂,尤其在重复单元之间存在差异时。本文目前采取的方案是寻找尽可能小的重复单元。这一方面可以减少表示歧义,另一方面可以增加拷贝断点的数量,以提升第二阶段识别潜在拷贝变异事件的灵敏度。

\section{识别序列内部重复}

本项目采用序列自比对的方式实现序列内部重复区域识别。区别于传统的Smith-Waterman(SW)模型,本方法在动态规划过程中插入了序列复制事件,即插入了一种新的转移路径。SW模型中允许匹配(错配)、插入、删除三种事件,对应“左上”、“左”和“上”三种转移路径。本方法额外允许序列复制事件,其转移路径来自于“右上”方的任意一个位置。在序列自比对过程中,SW矩阵的最优得分一定位于对角线上,且对角线两侧的得分完全镜像。为了驱使重复事件的得分远离对角线,重复事件必须获得额外的奖励(更小的罚分)。表 \ref{tab_parameters} 给出了本方案所使用的分值参数。

\begin{table}[h]
    \centering
    \caption{分值参数说明}
    \begin{tabular}{l|lrr}
        \hline
        \textbf{区域} & \textbf{参数} & \textbf{说明} & \textbf{默认值} \\
        \hline
        \multirow{4}{*}{非重复区域} & $A$ & 匹配得分 & 1 \\
        & $B$ & 错配罚分 & -4 \\
        & $G_o$ & 打开Gap罚分 & -6 \\
        & $G_e$ & 延伸Gap罚分 & -1 \\
        \hline
        \multirow{7}{*}{重复区域} & $u$ & 最小重复单元 & 5 \\
        & $d$ & 打开重复罚分 & -2 \\
        & $p$ & 关闭重复罚分 & -6 \\
        & $a$ & 匹配得分 & 2 \\
        & $b$ & 错配罚分 & -3 \\
        & $g_o$ & 打开Gap罚分 & -3 \\
        & $g_e$ & 延伸Gap罚分 & -1 \\
        \hline
    \end{tabular}
    \label{tab_parameters}
\end{table}

表 \ref{tab_parameters} 对重复区域和非重复区域采取了不同的打分方式,重复区域是由 $d$ 和 $p$ 所对应的事件标识的。$u$ 代表打开一个重复事件,从对角线向矩阵的左侧转移,亦称为 $D$ 转移。$p$ 代表关闭一个重复事件,即从矩阵的左侧向对角线转移,或称为 $B$ 转移。这两种转移事件必须有所罚分,否则最优结果将是在对角线和任意一个匹配之间的反复跳跃。尤其是 $B$ 转移的罚分,对识别连续的重复单元有深远的影响。介于 $D$ 和 $B$ 之间重复区域的打分比非重复区域的打分更加宽容,这样才能驱动重复事件的发生。此外,表 \ref{tab_parameters} 还对最小重复单元的大小做出限制,以防止由随机碰撞所产生的伪重复事件(如连续相同的几个字符)。

基于以上打分策略,下文给出本方法的动态规划递推式。为了更好地表示规则,递推式被划分为多个模块。

\[
j < i
\left\{
\begin{aligned}
    E_{i,j} & = \max(E_{i-1,j}, H_{i-1,j} + G_o) + G_e \\
    F_{i,j} & = \max(F_{i,j-1}, H_{i,j-1} + G_o) + G_e \\
    M_{i,j} & = H_{i-1,j-1} + (A \text{ or } B) \\
    H_{i,j} & = \max(E_{i,j}, F_{i,j}, M_{i,j}) \\
    D_{i,j} & = \max(H_{i-1,i-1} + d, h_{i-1,k}) \quad k \in {1..i-u} \\
\end{aligned}
\right.
\]

该部分描述了非对角线单元格的递归式,其中 $D$ 表示允许复制事件发生的通道。它可以由对角线转移而来,也可以由上一次重复转移而来。前者存在 $d$ 的罚分,而后者没有罚分。即只有第一次重复时罚分,后续的重复无需额外开销,以此鼓励重复事件形成更加规则的zigzag折线。

\[
j < i
\left\{
\begin{aligned}
    DF_{i,j} & = i-1 & \text{if } D_{i,j} = H_{i-1,i-1} + d \\
    CF_{i,j} & = j & \text{if } D_{i,j} = H_{i-1,i-1} + d \\
    CF_{i,j} &= CF_{i-1, k} & \text{if } D_{i,j} = \max(h_{i-1,k}) \\
\end{aligned}
\right.
\]

该部分通过 $CF$ 和 $DF$ 约束后续的重复比对的起始和结束位置,即 $D$ 转移后的重复匹配不应该超过对应重复单元所对应的范围。该方法避免了 $D$ 转移后无法再折回重复开头的情况,以及无意义的重复匹配。

\[
j < i
\left\{
\begin{aligned}
    e_{i,j} & = \max(D_{i-1,j}, e_{i-1,j}, h_{i-1,j} + g_o) + g_e \\
    f_{i,j} & = \max(D_{i,j-1}, f_{i,j-1}, h_{i,j-1} + g_o) + g_e \\
    m_{i,j} & = \max(h_{i-1,j-1}, D_{i-1,j-1}) + (A \text{ or } B) \quad j \in [CF_{i-1,j-1}, DF_{i-1,j-1}]\\
    h_{i,j} & = \max(e_{i,j}, f_{i,j}, m_{i,j}) \\
    D_{i,j} & = \max(D_{i,j}, h_{i,j}) \\
\end{aligned}
\right.
\]

该部分给出 $D$ 通道打开之后的重复事件匹配子矩阵。该子矩阵受到 $CF$ 和 $DF$ 的约束,而且每个单元格继承来自于其最优路径的 $CF$ 和 $DF$ 值。为了简化转移矩阵的表示形式,此处没有详细列出规则。在该子矩阵完成匹配后,有一次再次打开 $D$ 通道的机会,得分最高的通道门将会开启下一次重复子矩阵。

\[
j = i
\left\{
\begin{aligned}
    F_{i,j} & = \max(F_{i,j-1}, H_{i,j-1} + G_o) + G_e \\
    M_{i,j} & = H_{i-1,j-1} + (A \text{ or } B) \\
    H_{i,j} & = \max(E_{i,j}, F_{i,j}, M_{i,j}, h_{i,k} + p) \quad k \in {1..i-u} \\
\end{aligned}
\right.
\]

该部分给出对角线元素的转移方程。对角线单元格可以标识重复事件的结束,由重复子矩阵中的最大值转移而来。本方法仅计算全局矩阵中的对角线($j=i$)和左下三角阵($j<i$),右上三角阵($j>i$)不需要考虑。

以上即是本文算法识别序列内部串联重复事件的理论描述。虽然递推式比原始SW算法复杂很多,但理论时间复杂度并没有增加。若序列长度为 $N$,该算法的时间复杂度为 $O(N^2)$。

\newpage
\section{模拟数据测试}
本节在包含不同重复的模拟数据上对算法进行测试。表 \ref{tab_sim} 给出了测试数据的主要特征,后续中的每一页给出程序的输出以及可视化结果。

\begin{table}[h]
    \centering
    \caption{模拟数据测试}
    \begin{tabular}{lrrrr}
        \hline
        \textbf{样例} & \textbf{重复单元 [bp]} & \textbf{重复次数} & \textbf{变异率 [\%]} & \textbf{非重复长度 [bp]} \\
        \hline
        1 & 3 & 15 & 0 & 8, 8 \\
        2 & 5 & 10 & 0 & 20,20 \\
        3 & 3,5 & 15,10 & 0 & 8,20,8,20 \\
        4 & 10 & 20 & 3 & 80, 80 \\
        5 & 50 & 40 & 5 & 150, 125 \\
        6 & 100 & 100 & 5 & 1000,1000 \\
        \hline
    \end{tabular}
    \label{tab_sim}
\end{table}


\newpage
样例1通过,短于最小长度的重复仍可正确识别,且识别的长度尽可能短(识别结果长度为6,而非9或更大)。

\begin{center}
\begin{tcolorbox}[title=Case 1, width=0.8 \textwidth]
\begin{verbatim}
motif=CAC, period=15, mutation=0, flank=(8,8)
motif_len=3, seq_len=61
    Close: 13 -> 55 (Diagonal)
    Copy: 10 (i=52) -> 15, unit_size=6
    Copy: 10 (i=46) -> 15, unit_size=6
    Copy: 10 (i=40) -> 15, unit_size=6
    Copy: 10 (i=34) -> 15, unit_size=6
    Copy: 10 (i=28) -> 15, unit_size=6
    Copy: 10 (i=22) -> 15, unit_size=6
    Open: 10 (Diagonal) -> 15, unit_size=6
    Found 7 duplications
\end{verbatim}
\end{tcolorbox}
\end{center}

\begin{figure}[h]
    \centering
    \includegraphics[width=0.8 \textwidth]{self_1.png}
    \caption{Case 1}
\end{figure}


\newpage
样例2通过。

\begin{center}
\begin{tcolorbox}[title=Case 2, width=0.8 \textwidth]
\begin{verbatim}
motif=CCCAA, period=10, mutation=0, flank=(20,20)
motif_len=5, seq_len=90
    Close: 25 -> 70 (Diagonal)
    Copy: 21 (i=66) -> 25, unit_size=5
    Copy: 21 (i=61) -> 25, unit_size=5
    Copy: 21 (i=56) -> 25, unit_size=5
    Copy: 21 (i=51) -> 25, unit_size=5
    Copy: 21 (i=46) -> 25, unit_size=5
    Copy: 21 (i=41) -> 25, unit_size=5
    Copy: 21 (i=36) -> 25, unit_size=5
    Copy: 21 (i=31) -> 25, unit_size=5
    Open: 21 (Diagonal) -> 25, unit_size=5
    Found 9 duplications
\end{verbatim}
\end{tcolorbox}
\end{center}

\begin{figure}[h]
    \centering
    \includegraphics[width=0.8 \textwidth]{self_2.png}
    \caption{Case 2}
\end{figure}


\newpage
样例3包含两次重复事件,测试通过。

\begin{center}
\begin{tcolorbox}[title=Case 3, width=0.8 \textwidth]
\begin{verbatim}
Found 18 duplications
\end{verbatim}
\end{tcolorbox}
\end{center}

\begin{figure}[h]
    \centering
    \includegraphics[width=0.8 \textwidth]{self_3.png}
    \caption{Case 3}
\end{figure}


\newpage
样例4中重复单元开始包含较多差异,重复单元的连续性被打断。此处暴露出一个问题:表 \ref{tab_parameters} 的关闭重复罚分 $p$ 难以设置。当重复区域由于差异事件导致罚分时,算法倾向通过 $B$ 转移来逃避矩阵内部的惩罚。毕竟目前 $p$ 的分值为-6,当连续两个以上错配发生时(罚分为-9),动态规划的最优路径便成了图中所示的结果。

我之前所提到的问题正是对应这种情况:零花费从后向前转移,导致全局路径错误。若 $(i_2, j_2)$ 从 $(i_1, j_1)$ 转移而来,矩阵中可能存在介于二者之间的 $(i_3, j_3)$ 满足 $(i_1, j_1) \rightarrow (i_3, j_3) + (i_3, j_3) \rightarrow (i_2, j_2) > (i_1, j_1) \rightarrow (i_2, j_2)$。但在本文的理论模型下,该路径是正确的最优路径。

\begin{center}
\begin{tcolorbox}[title=Case 4, width=0.8 \textwidth]
\begin{verbatim}
motif=ACTTTCGAAG, period=20, mutation=6, flank=(80,80)
motif_len=10, seq_len=360
Found 19 duplications
\end{verbatim}
\end{tcolorbox}
\end{center}

\begin{figure}[h]
    \centering
    \includegraphics[width=0.8 \textwidth]{self_4.png}
    \caption{Case 4}
\end{figure}


\newpage
在样例4中,我大幅加重关闭重复的罚分 $p$,将其设置为-20。虽然程序可以输出期望的结果,但这会导致小规模的重复识别不出来。在面临情况未知的新数据时,我们无法通过手动调整参数以期待获得理想的结果。

\begin{center}
\begin{tcolorbox}[title=Case 4: p(-20), width=0.8 \textwidth]
\begin{verbatim}
motif=ACTTTCGAAG, period=20, mutation=6, flank=(80,80)
motif_len=10, seq_len=360
Found 17 duplications
\end{verbatim}
\end{tcolorbox}
\end{center}

\begin{figure}[h]
    \centering
    \includegraphics[width=0.8 \textwidth]{self_4_p20.png}
    \caption{Case 4: p=-20}
\end{figure}


\newpage
随着测试数据规模增大,错误数不断增加,该问题变得更加明显。但算法仍然能良好区分重复区域和非重复区域,侧面验证了逻辑实现的正确性。

\begin{center}
\begin{tcolorbox}[title=Case 5, width=0.8 \textwidth]
\begin{verbatim}
motif_length=40, period=40, mutation=90, flank=(150,125)
motif_len=50, seq_len=2284
Found 56 duplications
\end{verbatim}
\end{tcolorbox}
\end{center}

\begin{figure}[h]
    \centering
    \includegraphics[width=0.8 \textwidth]{self_5.png}
    \caption{Case 5}
\end{figure}


\newpage
此处给出 $10^4$ 规模级别数据的测试结果。在我的个人笔记本上测试,程序可以在数秒内输出结果。

\begin{center}
\begin{tcolorbox}[title=Case 6, width=0.8 \textwidth]
\begin{verbatim}
motif_length=100, period=100, mutation=96, flank=(1000,1000)
motif_len=100, seq_len=11995
Found 143 duplications
\end{verbatim}
\end{tcolorbox}
\end{center}

\begin{figure}[h]
    \centering
    \includegraphics[width=0.8 \textwidth]{self_6.png}
    \caption{Case 6}
\end{figure}


\newpage
\section{讨论}
\begin{enumerate}[itemsep=0pt, topsep=0pt, label=(\arabic*)]
    \item 本方法在复杂数据上可以高效识别重复事件,但目前 $B$ 转移并不能准确识别重复事件的结束。
    \item 本方法依赖复杂,加速难度更高。
    \item 第一阶段的核心目标是识别潜在的重复事件位点,该过程其实可以由时间复杂度为 $O(N\log{N})$ 的分治算法实现[\ref{gad_2001}, \ref{michael_1984}],这将大幅增加算法的效率和可优化空间。
\end{enumerate}

\section{参考文献}
\begin{enumerate}
    \item Gary Benson. 1997. Sequence alignment with tandem duplication. \label{gary_1997}
    \item Vincent A. Fischetti, Gad M. Landau, Jeanette P. Schmidt and Peter H. Sellers. 1992. Identifying periodic occurrences of a template with applications to protein structure. \label{wraparound}
    \item Gad M. Landau, Jeanette P. Schmidt and Dina Sokol. 2001. An algorithm for approximate tandem repeats. \label{gad_2001}
    \item Michael G. Main and Richard J. Lorentz. 1984. An O (n log n) algorithm for finding all repetitions in a string. \label{michael_1984}
\end{enumerate}

\end{document}